\section{Discussion}
Looking back on the results, one could say that statistically the answers found on the two question hold, but of course there is a lot to say about the depth of this analysis and if the results are generally applicable in the real world. For example, let's look at the level of detail of the data sets used in the analysis. The data frames used are condensed data sets. Look at the data set with the average house price. This set holds the average price based on thousands of transactions. For more in depth analyses and better understanding of the results, it might be better to use the underlying data set of all these transactions and not the condensed average. The same holds for the set of restaurants. The correlation is now based on the number of restaurants in the main or capital city of a province. Taking just the restaurants of the mid point of the province was done due to the limitations of the free requests possible to FourSquare. If we would use a data set with all venues instead of just a this subset, the correlation could turn out differently. Last but not least, just looking at restaurants being listed in FourSquare, without examining the venues in more detail like looking at reviews and actual use (=number of guests per annum) of these venues, leaves a lot of room for improvement on the thoroughness of the analysis.